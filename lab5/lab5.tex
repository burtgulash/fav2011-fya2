\documentclass[11pt]{article}
\usepackage[utf8]{inputenc}
\usepackage[czech]{babel}
\usepackage{a4wide}
\usepackage{graphicx}

\newcommand{\labtitle}[4]{
	\begin{titlepage}
		\begin{center}
			\mbox{} \\[4cm]
			{\huge {#1}} \\[2cm]
			{\Large #2}  % \\[.2cm]
			{\large #3 } \\[.7cm]
			{\normalsize měřeno #4}
		\end{center}
	\end{titlepage}
}


\begin{document}

% -- title page -- %
\labtitle{Debyeova-Scherrerova metoda určení jemné struktury materiálu}
 {Tomáš Maršálek}
 {(A10B0632P)}
 {28.\,listopadu 2011}
% -- title page -- %

\section{Měřící potřeby a přístroje}
exponovaný rentgenový snímek, zařízení pro odečítání reflexí ze snímku

\section{Naměřené hodnoty}
\begin{center}
\begin{tabular}{|c|c|c|c|c|c|}
\hline
$2\theta$ & $\theta$ & $\sin\theta$ & d [\AA] & h k l & a [\AA] \\
\hline
0.522 & 0.261 & 0.2580 & 3.468 & 1 1 1 & 6.007 \\
0.616 & 0.308 & 0.3031 & 2.952 & 0 0 2 & 5.904 \\
0.889 & 0.444 & 0.4300 & 2.081 & 2 0 2 & 5.886 \\
1.064 & 0.532 & 0.5072 & 1.764 & 1 1 3 & 5.850 \\
1.124 & 0.562 & 0.5328 & 1.679 & 2 2 2 & 5.816 \\
1.334 & 0.667 & 0.6186 & 1.446 & 0 0 4 & 5.784 \\
1.491 & 0.745 & 0.6783 & 1.319 & 3 3 1 & 5.749 \\
1.540 & 0.770 & 0.6961 & 1.285 & 4 2 0 & 5.746 \\
1.736 & 0.868 & 0.7630 & 1.173 & 4 2 2 & 5.747 \\
1.908 & 0.954 & 0.8157 & 1.097 & 5 1 1 & 5.700 \\
2.184 & 1.092 & 0.8875 & 1.008 & 4 4 0 & 5.702 \\
2.408 & 1.204 & 0.9334 & 0.958 & 5 3 1 & 5.668 \\
2.482 & 1.241 & 0.9461 & 0.946 & -     & -     \\
\hline
\end{tabular}
\end{center}
\section{Výpočty}
\section{Závěr} 
\end{document}










































